% Created 2021-11-07 dom 22:10
% Intended LaTeX compiler: pdflatex
\documentclass[xcolor=dvipsnames, 10pt, presentation,aspectratio=169]{beamer}
\usepackage[utf8]{inputenc}
\usepackage[T1]{fontenc}
\usepackage{graphicx}
\usepackage{grffile}
\usepackage{longtable}
\usepackage{wrapfig}
\usepackage{rotating}
\usepackage[normalem]{ulem}
\usepackage{amsmath}
\usepackage{textcomp}
\usepackage{amssymb}
\usepackage{capt-of}
\usepackage{hyperref}
\setbeamertemplate{footline}[frame number]
\usecolortheme[named=BrickRed]{structure}
\setbeamertemplate{navigation symbols}{}
\usepackage[american, english]{babel}
\usepackage{url} \urlstyle{sf}
\useinnertheme{circles}
\let\alert=\structure
\usepackage{wrapfig}
\usepackage{fancyvrb}
\newcommand{\bashcmd}[1]{\textcolor{White}{\colorbox{Sepia}{\texttt{#1}}}}
\logo{ \includegraphics[height=1cm,width=1cm,keepaspectratio]{logo_inf}    \includegraphics[height=1cm,width=1cm,keepaspectratio]{logo_ufsm} }
\usetheme{Madrid}
\author{João Vicente Ferreira Lima}
\date{2021/2}
\title{Bash Scripting}
\subtitle{Operating System Practice}
\institute[UFSM]{Universidade Federal de Santa Maria \\ \url{jvlima@inf.ufsm.br} \\ \url{http://www.inf.ufsm.br/~jvlima}}
\hypersetup{
 pdfauthor={João Vicente Ferreira Lima},
 pdftitle={Bash Scripting},
 pdfkeywords={},
 pdfsubject={},
 pdfcreator={Emacs 24.5.1 (Org mode 8.3.4)}, 
 pdflang={English}}
\begin{document}

\maketitle
\frame<handout:0>
{
  \frametitle{Outline}
  \tableofcontents
}

\makeatletter
\AtBeginSubsection[]
{
  \frame<handout:0>
  {
    \frametitle{Outline}
    \tableofcontents[current,currentsubsection]
  }
}
\makeatother

\section{Bash Scripting}
\label{sec:org39a313b}
\subsection{Arithmetic Operations}
\label{sec:org55f1e2f}
\begin{frame}[label={sec:org5d02e79},fragile]{Arithmetic operations}
 \begin{block}{}
\begin{verbatim}
#!/bin/bash
# Counting to 11 in 10 different ways.

n=1; echo -n "$n "

let "n = $n + 1"   # let "n = n + 1"  also works.
echo -n "$n "

\end{verbatim}
\end{block}
\pause
\begin{alertblock}{No floating point}
Bash does not understand floating point arithmetic. It treats numbers
containing a decimal point as strings.
\end{alertblock}
\end{frame}
\begin{frame}[label={sec:org715b9c8},fragile]{Arithmetic operations}
 \begin{block}{}
\begin{verbatim}
: $((n = $n + 1))
#  ":" necessary because otherwise Bash attempts
#+ to interpret "$((n = $n + 1))" as a command.
echo -n "$n "

(( n = n + 1 ))
#  A simpler alternative to the method above.
echo -n "$n "

n=$(($n + 1))
echo -n "$n "
\end{verbatim}
\end{block}
\end{frame}
\begin{frame}[label={sec:org6d51783},fragile]{Arithmetic operations}
 \begin{block}{}
\begin{verbatim}
: $[ n = $n + 1 ]
#  ":" necessary because otherwise Bash attempts
#+ to interpret "$[ n = $n + 1 ]" as a command.
#  Works even if "n" was initialized as a string.
echo -n "$n "

n=$[ $n + 1 ]
#  Works even if "n" was initialized as a string.
#* Avoid this type of construct, since it is obsolete and nonportable.
echo -n "$n "
\end{verbatim}
\end{block}
\end{frame}
\begin{frame}[label={sec:org0a7df23},fragile]{Arithmetic operations}
 \begin{block}{}
\begin{verbatim}
# Now for C-style increment operators.
let "n++"          # let "++n"  also works.
echo -n "$n "

(( n++ ))          # (( ++n ))  also works.
echo -n "$n "

: $(( n++ ))       # : $(( ++n )) also works.
echo -n "$n "

: $[ n++ ]         # : $[ ++n ] also works
echo -n "$n "

exit 0
\end{verbatim}
\end{block}
\end{frame}
\subsection{Variables}
\label{sec:orgadc84a1}
\begin{frame}[label={sec:orgcc303ab}]{Internal variables}
Builtin variables:
\begin{itemize}
\item \bashcmd{\$BASH} path to the Bash itself
\item \bashcmd{\$BASH\_VERSION} version of Bash
\item \bashcmd{\$EDITOR} the default editor
\item \bashcmd{\$HOME} home directory
\item \bashcmd{\$PATH} path to binaries
\item \bashcmd{\$PWD} working directory
\item \bashcmd{\$UID} user ID number
\end{itemize}
Positional variables:
\begin{itemize}
\item \bashcmd{\$?} exit status  of a command
\item \bashcmd{\$\$} process ID (PID)
\end{itemize}
\end{frame}
\begin{frame}[label={sec:orgb441af1},fragile]{Random}
 \bashcmd{RANDOM} is a internal Bash function that returns a
\emph{pseudorandom} integer in 0 - 32767.
\begin{block}{}
\begin{verbatim}
RANDOM=$$      #  Seeds the random number generator from PID
               #+ of script.

for i in $(seq 1 10)
do
    echo $RANDOM
done
\end{verbatim}
\end{block}
\end{frame}
\subsection{Strings}
\label{sec:org95552b8}
\begin{frame}[label={sec:org794198d},fragile]{Manipulating strings}
 \alert{String length} - \texttt{\$\{\#string\}}
\begin{block}{}
\begin{verbatim}
stringZ=abcABC123ABCabc

echo ${#stringZ}     # 15
\end{verbatim}
\end{block}
\alert{Substring extraction} - \texttt{\$\{string:position:length\}}
\begin{block}{}
\begin{verbatim}
stringZ=abcABC123ABCabc
echo ${stringZ:7}            # 23ABCabc
echo ${stringZ:7:3}          # 23A
                             # Three characters of substring.
\end{verbatim}
\end{block}
\end{frame}
\begin{frame}[label={sec:org740c489},fragile]{Manipulating strings}
 \begin{block}{Random password}
\begin{verbatim}
if [ -n "$1" ]  #  If command-line argument present,
then            #+ then set start-string to it.
  str0="$1"
else            #  Else use PID of script as start-string.
  str0="$$"
fi
POS=2  # Starting from position 2 in the string.
LEN=8  # Extract eight characters.

str1=$( echo "$str0" | md5sum | md5sum )
#  Doubly scramble     ^^^^^^   ^^^^^^

randstring="${str1:$POS:$LEN}"

echo "$randstring"
\end{verbatim}
\end{block}
\end{frame}
\begin{frame}[label={sec:orgb2b7ff4},fragile]{Manipulating strings}
 \alert{Substring removal} - \texttt{\$\{string\#substring\}} deletes shortest match,
 \texttt{\$\{string\#\#substring\}} deletes longest match.
\begin{block}{}
\begin{verbatim}
stringZ=abcABC123ABCabc
#       |----|          shortest
#       |----------|    longest

echo ${stringZ#a*C}      # 123ABCabc
# Strip out shortest match between 'a' and 'C'.

echo ${stringZ##a*C}     # abc
# Strip out longest match between 'a' and 'C'.
\end{verbatim}
\end{block}
\end{frame}
\begin{frame}[label={sec:org6aa6262},fragile]{Manipulating strings}
 \alert{Substring replacement} - \texttt{\$\{string/substring/replacement\}}, replace first
 \emph{match} of \texttt{\$substring} with \texttt{\$replacement}.

\alert{Substring replacement} - \texttt{\$\{string//substring/replacement\}}, replace all
 matches of \texttt{\$substring} with \texttt{\$replacement}.
\begin{block}{}
\begin{verbatim}
stringZ=abcABC123ABCabc

echo ${stringZ/abc/xyz}  # xyzABC123ABCabc
              # Replaces first match of 'abc' with 'xyz'.

echo ${stringZ//abc/xyz} # xyzABC123ABCxyz
              # Replaces all matches of 'abc' with # 'xyz'.
\end{verbatim}
\end{block}
\end{frame}
\subsection{Parameter substitution}
\label{sec:orgeeea523}
\begin{frame}[label={sec:org8e9e604},fragile]{Parameter substitution}
 Manipulating and/or expanding variables:
\begin{itemize}
\item \texttt{\$\{parameter\}}
\item \texttt{\$\{parameter-default\}} if parameter not set, use \texttt{default}.
\end{itemize}
\begin{block}{}
\begin{verbatim}
echo ${username-`whoami`}
# Echoes the result of `whoami`, if variable $username 
# is still unset.

DEFAULT_FILENAME=generic.data
filename=${1-$DEFAULT_FILENAME}
#  if parameter $1 is not specified
\end{verbatim}
\end{block}
\end{frame}
\begin{frame}[label={sec:orgcc5968a},fragile]{Parameter substitution}
 \texttt{\$\{parameter=default\}} - If parameter not set, set it to \emph{default}.
\begin{block}{}
\begin{verbatim}
echo ${var=abc}   # abc
echo ${var=xyz}   # abc
# $var had already been set to abc, so it did not change.
\end{verbatim}
\end{block}
\end{frame}
\subsection{Loops}
\label{sec:org77be454}
\begin{frame}[label={sec:org9f1571e},fragile]{for}
 \begin{block}{Parameterized file list}
\begin{verbatim}
#!/bin/bash

filename="*txt"

for file in $filename
do
 echo "Contents of $file"
 echo "---"
 cat "$file"
 echo
done
\end{verbatim}
\end{block}
\end{frame}
\begin{frame}[label={sec:orgf818c64},fragile]{for}
 \begin{block}{File expansion}
\begin{verbatim}
#!/bin/bash
# Globbing = filename expansion.

for file in *
#           ^  Bash performs filename expansion
#+             on expressions that globbing recognizes.
do
    if [ -d "$file" ]; then
        echo "$file is a directory"
    fi
    if [ -f "$file" ]; then
        echo "$file is a regular file."
    fi
done

exit 0
\end{verbatim}
\end{block}
\end{frame}
\begin{frame}[label={sec:org1490080},fragile]{for}
 \begin{block}{Function}
\begin{verbatim}
generate_list ()
{
  echo "one two three"
}

for word in $(generate_list)  # Let "word" grab output of function.
do
  echo "$word"
done
\end{verbatim}
\end{block}
\end{frame}
\begin{frame}[label={sec:orge6e5fc3},fragile]{for}
 \begin{block}{Counting to ten}
\begin{verbatim}
# Using "seq" ...
for a in `seq 10`
do
  echo -n "$a "
done  

echo; echo
\end{verbatim}
\end{block}
\end{frame}
\begin{frame}[label={sec:orgb085198},fragile]{for}
 \begin{block}{Counting to ten}
\begin{verbatim}
# Now, let's do the same, using C-like syntax.

LIMIT=10
# Double parentheses, and naked "LIMIT"
for ((a=1; a <= LIMIT ; a++))
do
  echo -n "$a "
done

echo; echo
\end{verbatim}
\end{block}
\end{frame}
\begin{frame}[label={sec:org08060e2},fragile]{while}
 \begin{block}{While to ten}
\begin{verbatim}
#!/bin/bash

var0=0
LIMIT=10

while [ "$var0" -lt "$LIMIT" ]
do
  echo -n "$var0 "        # -n suppresses newline.

  var0=$(($var0+1))
done

echo
exit 0
\end{verbatim}
\end{block}
\end{frame}
\begin{frame}[label={sec:orgc48139c},fragile]{while}
 \begin{block}{Test to end}
\begin{verbatim}
#!/bin/bash
                           # Equivalent to:
while [ "$var1" != "end" ] # while test "$var1" != "end"
do
  echo "Input variable #1 (end to exit) "
  read var1                   # Not 'read $var1' (why?).
  echo "variable #1 = $var1"  # Need quotes because of "#" . . .
  # If input is 'end', echoes it here.
  echo
done  

exit 0
\end{verbatim}
\end{block}
\end{frame}
\begin{frame}[label={sec:org84bf145},fragile]{while}
 \begin{block}{C-style while}
\begin{verbatim}
LIMIT=10      # 10 iterations.
((a = 1))      # a=1

while (( a <= LIMIT ))   #  Double parentheses,
do                       #+ and no "$" preceding variables.
  echo -n "$a "
  ((a += 1))             # let "a+=1"
done

echo
exit 0
\end{verbatim}
\end{block}
\end{frame}
\begin{frame}[label={sec:org631e4e6},fragile]{while}
 \begin{block}{While and pipes}
\begin{verbatim}
#!/bin/bash

ps aux | \
while read user pid cpu mem vsz rss tty stat start time command
do
    echo $pid $mem $command
done | sort -n -r -k2

# sorts by memory usage

exit 0
\end{verbatim}
\end{block}
\end{frame}
\begin{frame}[label={sec:orgd2531e5},fragile]{while}
 \begin{block}{Reading files}
\begin{verbatim}
#!/bin/bash

IFS=':' # internal field separator

while read account password uid gid gecos directory shell
do
    echo $uid $account
done < /etc/passwd

exit 0
\end{verbatim}
\end{block}
\end{frame}
\begin{frame}[label={sec:org83d089f},fragile]{until}
 This construct tests for a condition at the top of a loop, and keeps
looping as long as that condition is \emph{false} (opposite of \emph{while loop}).
\begin{block}{}
\begin{verbatim}
#!/bin/bash
LIMIT=10
var=0

until (( var > LIMIT ))
do
  echo -n "$var "
  (( var++ ))
done    # 0 1 2 3 4 5 6 7 8 9 10 


exit 0
\end{verbatim}
\end{block}
\end{frame}
\begin{frame}[label={sec:orgcaa410f},fragile]{Loop control}
 The \bashcmd{break} and \bashcmd{continue} loop control commands
correspond exactly to their counterparts in other programming
languages.
\begin{block}{}
\begin{verbatim}
LIMIT=19  # Upper limit
echo "Printing Numbers 1 through 20 (but not 3 and 11)."
a=0
while [ $a -le "$LIMIT" ]
do
 a=$(($a+1))

 if [ "$a" -eq 3 ] || [ "$a" -eq 11 ]; then
   continue
 fi

 echo -n "$a "   # This will not execute for 3 and 11.
done 
\end{verbatim}
\end{block}
\end{frame}

\subsection{Functions}
\label{sec:orgc87dd95}
\begin{frame}[label={sec:orge5eeb0e},fragile]{Functions}
 \begin{block}{Simple functions}
\begin{verbatim}
fun () { echo "This is a function"; echo; }

foo() {
    echo "foo"
}

fun
foo
\end{verbatim}
\end{block}
\end{frame}
\begin{frame}[label={sec:org1c52026},fragile]{Functions}
 \begin{block}{Arguments (1/2)}
\begin{verbatim}
#!/bin/bash
DEFAULT=default 
func2 () {
   if [ -z "$1" ]; then
     echo "-Parameter #1 is zero length.-" 
   else
     echo "-Parameter #1 is \"$1\".-"
   fi
   variable=${1-$DEFAULT} 
   echo "variable = $variable"
   if [ "$2" ]; then
     echo "-Parameter #2 is \"$2\".-"
   fi

   return 0
}
\end{verbatim}
\end{block}
\end{frame}
\begin{frame}[label={sec:orgb20e73e},fragile]{Functions}
 \begin{block}{Arguments (2/2)}
\begin{verbatim}
#!/bin/bash
echo "Two parameters passed."   
func2 first second  # Called with two params
echo

echo "\"\" \"second\" passed."
func2 "" second  # Called with zero-length first parameter
echo             # and ASCII string as a second one.

exit 0
\end{verbatim}
\end{block}
\end{frame}
\begin{frame}[label={sec:org6161f02},fragile]{Functions}
 Functions return a value, called an \emph{exit status}. This is analogous to
the exit status returned by a command.
\begin{block}{Exit status}
\begin{verbatim}

E_PARAM_ERR=250 # if no parameter
foo () {
   if [ -z "$1" ]; then
       return $E_PARAM_ERR
   fi
   return 0
}

foo ; res=$?
if [ "$res" -eq $E_PARAM_ERR ]; then
    echo "Missing parameter ..."
fi
exit 0
\end{verbatim}
\end{block}
\end{frame}
\begin{frame}[label={sec:org4985c66},fragile]{Functions}
 In contrast to C, a Bash variable declared inside a function is local
ONLY if declared as such.
\begin{block}{Global or local}
\begin{verbatim}
#!/bin/bash
func () {
  local loc_var=23       # Declared as local variable.
  global_var=999         # Not declared as local.
 }  
func

echo "\"loc_var\" outside function = $loc_var"

echo "\"global_var\" outside function = $global_var"

exit 0
\end{verbatim}
\end{block}
\end{frame}
\subsection{Arrays}
\label{sec:orgb03630b}
\begin{frame}[label={sec:orged92a1b},fragile]{Arrays}
 \begin{block}{Sparse arrays}
\begin{verbatim}
#!/bin/bash

area[11]=23
area[51]=UFOs

echo -n "area[11] = "
echo ${area[11]}    #  {curly brackets} needed.

echo "Contents of area[51] are ${area[51]}."

# Contents of uninitialized array variable print blank (null variable).
echo -n "area[43] = "
echo ${area[43]}
echo "(area[43] unassigned)"

\end{verbatim}
\end{block}
\end{frame}
\begin{frame}[label={sec:org49e7789},fragile]{Arrays}
 \begin{block}{}
\begin{verbatim}
#!/bin/bash
# Quoting permits embedding whitespace within individual 
#+ array elements.
array2=( [0]="first element" [1]="second element"
         [3]="fourth element" )

echo ${array2[0]}   # first element
echo ${array2[1]}   # second element
echo ${array2[2]}   # Skipped in initialization, and therefore null.
echo ${array2[3]}   # fourth element
echo ${#array2[0]}  # 13    (length of first element)
echo ${#array2[*]}  # 3     (number of elements in array)

exit
\end{verbatim}
\end{block}
\end{frame}
\begin{frame}[label={sec:org1d0b056}]{}
\end{frame}
\end{document}